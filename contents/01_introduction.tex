\section{Introduction}\label{sec:introduction}
\subsection{研究目的與動機}
隨著\textbf{腦機介面(Brain–Computer Interface, BCI)}技術的不斷演進,腦波訊號作為人機互動的媒介正逐漸從實驗室走向實際應用。BCI 系統能讓使用者不透過傳統輸入裝置(如鍵盤、滑鼠或觸控)即可與外部設備進行互動,對於身心障礙輔具、情緒監控、自適應學習與互動娛樂等領域具備高度潛力。\\

在眾多可能應用中,\textbf{「心理狀態的即時辨識」}是一項極具發展潛力的方向。透過腦波訊號判斷使用者是否處於放鬆、專注、壓力或記憶負荷等狀態,不僅可提升系統的自適應能力,也能為智慧助理、教育科技與數位心理健康輔助提供關鍵依據。\\

基於上述動機,我們希望設計一套具備實用性、即時性與互動性的完整 BCI 系統,不僅能準確判斷心理狀態,還能將結果即時應用於遊戲控制與語意回饋之中,實踐從「大腦感知」到「智慧互動」的完整流程。

\subsection{相關背景:EEG 與心理狀態分類之應用}
\textbf{腦電圖(Electroencephalography, EEG)}是目前 BCI 系統中最常見的訊號來源。它以非侵入方式量測大腦皮質表層的電活動,具備高時間解析度,能即時反映使用者的認知與情緒狀態。EEG 訊號中的特定頻段(如 Alpha、Beta、Theta、Gamma 等)常與特定心理狀態相關聯,例如 Alpha 波段與放鬆相關,Beta 波段與專注或壓力相關。 \\

透過適當的訊號前處理、特徵擷取與分類模型訓練,EEG 可被用來進行心理狀態分類。隨著深度學習與模型優化技術的進展,許多研究已成功將腦波訊號分為多個狀態,並應用於控制輪椅、機械手臂、遊戲或情緒輔助系統。然而,將離線模型有效移植至即時互動應用,仍是一項具挑戰性的任務,牽涉到系統延遲、資料品質波動與使用者參與穩定度等問題。

\subsection{專案目標與設計概述}
本專案分為三個任務(Task)進行,目標是建構一個可即時運作的 BCI 系統,並探索其在互動遊戲與智慧助理上的應用可能:

\begin{itemize}
    \item \textbf{Task 1:腦波分類模型訓練} \\
    我們使用多受試者 EEG 資料,訓練一個能辨識四種心理狀態\textbf{(放鬆 Relax、專注 Focus、壓力 Stress、記憶 Memory)}的一維分類模型,並透過 LOSO 驗證與 F1-score 評估模型表現。

\item \textbf{Task 2:即時腦波遊戲控制系統} \\
將 Task 1 訓練出的分類模型應用於即時情境,建立一個由 EEG 控制角色移動的迷宮遊戲系統,包含資料擷取、自動化前處理、即時預測與角色控制四個模組,實現真正的\textbf{ BCI 應用遊戲互動}。

\item \textbf{Task 3:Neuro Chat 專屬腦波分析師} \\
我們設計一款名為「Neuro Chat」 的腦波聊天機器人,\textbf{結合 EEG 分類模型、GPT 大型語言模型與 LINE Bot 前端,打造一個能夠「讀懂使用者心理狀態並給予即時回應與鼓勵」的智慧聊天助理}。使用者只需輸入文字,同時量測腦波訊號,即可讓 Neuro Chat 了解使用者的狀況,完成與 AI 的對話互動,提升人機交流的情感深度與使用便捷性。
\end{itemize}

本專案不僅展示 BCI 系統在分類與控制層面的技術實現,更進一步探索其與語意生成模型整合的可能性,試圖將腦波應用由功能性操作延伸至情感性互動,為未來 BCI 技術在智慧助理與情緒科技領域開創新方向。
